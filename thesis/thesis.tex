%	Oskar L F Lerivåg
%	Masters Thesis

\documentclass[a4paper]{article}
%Bookmarks
\usepackage[colorlinks=true,urlcolor=cyan,linkcolor=black,citecolor=red,bookmarksopen=true]{hyperref}
\usepackage{bookmark}

\usepackage{fancyref}
\usepackage[utf8]{inputenc}
\usepackage{amsmath}
\usepackage{amsthm}
\usepackage[framemethod=TikZ]{mdframed}
\usepackage{pgf,tikz}
\usepackage{mathrsfs}
\usepackage{listings}
\usepackage{amssymb}
\usepackage{url}
\usepackage{graphicx}

\usepackage{float}
\floatplacement{figure}{H} % forces figures to be placed at the correct location

\usepackage{enumerate} % Needed for markdown enumerations to work

\usepackage[mathletters]{ucs}

\usepackage{titling} %Remove insane whitespaces

%Margins
\usepackage{geometry}
\geometry{a4paper, margin=3cm}

%Algorithms
\usepackage{algorithm, algorithmicx}
\usepackage{algpseudocode}
\newcommand{\myFigure}[3]{\begin{figure}[h!]\centering\includegraphics[scale=#1]{figures/#2}\caption{#3}\end{figure}}

\newtheorem{theoremdefinition}{\textbf{Definition}}

\algnewcommand\algorithmicforeach{\textbf{foreach}}
\algdef{S}[FOR]{ForEach}[1]{\algorithmicforeach\ #1\ \algorithmicdo}

\algdef{S}[FOR]{ForTo}[2]{\algorithmicfor\ #1 \textbf{to} #2\ \algorithmicdo}

\newenvironment{problem}[2][]{
   \mdfsetup{
        frametitle={
            \tikz[baseline=(current bounding box.east),outer sep=0pt]
            \node[anchor=east,rectangle,fill=white!20]
            {\strut ~#1};}
        }
    
    \mdfsetup{
        innertopmargin=10pt,linecolor=black!30,
        linewidth=1pt,topline=true,
        roundcorner=5pt,
        frametitleaboveskip=\dimexpr-\ht\strutbox\relax
    }
 
\begin{mdframed}[]\relax}{
\end{mdframed}}

\begin{document}

% % % % % % % % % % % % % % % % %
%
%	FRONT PAGE
%
\begin{titlepage}
    \newcommand{\HRule}{\rule{\linewidth}{0.5mm}}
    \center

    \textsc{\LARGE University of Bergen}\\[1.5cm] % Name of your university/college
    \textsc{\large Masters thesis of Informatics:Algorithms}\\[0.5cm] % Minor heading such as course title

    \HRule \\[0.4cm]
    { \huge \bfseries Practical implementation of a parametrized binary matrix clustering algorithm}\\[0.4cm] % Title of your document
    \HRule \\[1.5cm]

    \Large \emph{Student:}\\
    Oskar L. F. Leirvåg\\
    \Large \emph{Mentor:}\\
	Petr A. Golovach
    \\[2cm] % Your name

    \centerline{\includegraphics[scale=0.5]{figures/canvas}} % Include a department/university logo - this will require the graphicx package

    {\large \today}\\[3cm] % Date, change the \today to a set date if you want to be precise

    \vfill
\end{titlepage}


% % % % % % % % % % % % % % % % %
%
%	Acknowledgements
%
\section*{Acknowledgements}
\newpage

% % % % % % % % % % % % % % % % %
%
%	0. Abstract
%
\section*{Abstract}
\newpage

% % % % % % % % % % % % % % % % %
%
%	TABLE OF CONTENTS
%
\pdfbookmark{\contentsname}{toc}
\tableofcontents
\newpage

% % % % % % % % % % % % % % % % %
%
%	TABLE OF ALGORITHMS
%
\addcontentsline{toc}{section}{List of Algorithms}
\listofalgorithms
\newpage


%%%%%%%%%%%%%%%%%%%%%%%%%%%%%%%%%%%%%%%%%%%%%%%%%%%%%%%%%%%%%%%%%%%%%%%%%%%%%%%%%%%%%%%%%%
%%%%%%%%%%%%%%%%%%%%%%%%%%%%%%%%%%%%%%%%%%%%%%%%%%%%%%%%%%%%%%%%%%%%%%%%%%%%%%%%%%%%%%%%%%

% % % % % % % % % % % % % % % % %
%
%	1. Introduction
%
\section{Introduction}
Though clustering itself does not have a strict definition, it has been researched and used 
extensively. As any comparable data-point might be more or less similar to other components, 
in which we can measure the similarity, can also contain data in the relationship between the 
similarity to other comparable components; it has long been an idea to group comparable 
components into classifications based on the components variables. By pre-determining the 
classifications one also pre-determines which variables in a comparable component is to be 
evaluated. The goal of clustering is therefore to compare a relationship among all variables, 
and regroup the data such that the components that are in the same group, are more similar 
to each other than those in other groups. \cite{gan07}

\subsection{Problem statement}
\begin{problem}[Binary r-Means]{prob:rm}
    \begin{tabular}{p{0.1\textwidth}p{0.8\textwidth}}
    \textit{Input}: & An $n \times m$ matrix \textbf{A} with columns 
    ($\textbf{a}^1,...,\textbf{a}^n$), a positive integer $\textbf{r}$ and a nonnegative 
    integer $\textbf{k}$\\
    
    \textit{Task}: & Decide whether there is a positive integer $\textbf{r}\sp{\prime} \leq 
    \textbf{r}$, a partition $\{I_1, ..., I_{r\sp{\prime}}\}$ of $\{1,...,n\}$ and vectors 
    $(\textbf{c}^1,...,\textbf{c}^{r\sp{\prime}}) \in \{0,1\}^m$ such that
    
    \[
        \sum_{i = 1}^{r\sp{\prime}} \sum_{j \in I_i} d_H(c^i, a^j) \leq k
    \]
    \end{tabular}
\end{problem}

\subsection{Example}
If a streaming webpage tracks whether their users like any given movie or not, they might want 
to use this data to recommend movies to any specific user which they might like. This is an
example of when we can create an approximation matrix holding centres of which any user would
be most likely to share the same movie interest with. In our example we assume that all the
included users rated all movies.


If we create an initial matrix of users and movies where we place all users as individual 
columns where the vector position indicate whether a user liked the movie ('1') or disliked the 
movie ('0'), and run the clustering algorithm, then the resulting centres list would be the 
new recommendations. All that remains is to find the closest centre for any new users vector,
and update anytime the users vector changes. 


When running the algorithm we also have to select the parameters $k$ and $r$ where $r$ would be
how many centres of movie-personalities we wish to define, and $k$ how much the users in
the training-set would be allowed to deviate from the centre.

\subsection{Binary cluster}
\subsection{Categorical cluster}

\subsection{Motivation}
Much research has gone into developing theoretical algorithms for the most complex computational 
problems in computer science, however many of them remain purely theoretical, not because 
they're not useful, but because the implementation of brute-force algorithms is simpler, and 
computational power can be bought. Testing the viability and practicality of implementing 
these complex algorithms is an important step in the research done for improving the 
algorithms, and the many practical limitations we face while constructing them in our 
preferred languages.


This paper aims to clarify the issues faced during development, and the results one could expect 
on a practical level of computation time on a smaller home computer/laptop.

\subsection{Previous results}
empty

% % % % % % % % % % % % % % % % %
%
%	2. Preliminaries
%
\newpage

\section{Preliminaries}
\subsection{Notations and Definitions}
\subsubsection{Matrices}
We here use matrices to define any $n \times m$ matrix $A$ where $n$ is the number of columns 
and $m$ is the number of rows, where $\forall A(n,m) \in \{0,1\}$.


It is also useful to note a column of a matrix $A$ as vector $a_n = (a_1,...a_m)^T$, or the row
of a matrix $A$ as $a_m = (a_1,...,a_n)$.


For clustering we need to compare the columns of matrices by distance, to calculate the distance
between columns we will use the hamming distance. Let us recall that the hamming distance
between two binary vectors $x,y \in \{0,1\}^m$ where $x=\{x_1,...,x_m\}^T$ and 
$y=\{y_1,...,y_m\}^T$, is 
\[
    d_H(x,y)= \sum_{i = 1}^{m} |x_i - y_i|
\]
or, in other words, the 
number of positions $i \in \{1,...,m\}$ where $x_i$ and $y_i$ differ. With that distance we can
also define the distance between two matrices as the sum of the distance between the columns of
the matrices.


With this lets also define the cardinality of a binary vector as the number of positive bits in 
the vector $v \in {0,1}^m$ such that $d_C(v)=\sum_{i = 1}^{m} v^i$. As the binary XOR ($\otimes$) 
operation between two bits results $i \otimes j = |i - j|$, we see that hamming distance between 
the two binary vectors can be defined
\[
    d_H(x, y) = d_C(x \otimes y) =\sum_{i = 1}^{m} (x^i \otimes y^i) = \sum_{i = 1}^{m} |x_i - y_i|
\]


For two binary $n \times m$ matrices with columns $A=(a^1,...,a^n)$ and 
$B=(b^1,...,b^n)$, the hamming distance we define as
\[
    d_H(A,B) = \sum_{i = 1}^{n} d_H(a^i, b^i)    
\]

\subsection{Parameterized algorithms}
\subsubsection{FPT}
\subsubsection{Kernel}
\subsubsection{Turing kernel}

\subsection{Problem complexity / kernelization}
\subsection{Turing kernels}

% % % % % % % % % % % % % % % % %
%
%	3. Algorithm
%
\newpage

\section{Algorithm}
\subsection{Kernelization}
\begin{algorithm}[H]
    \caption{Pre-processing for Binary r-Means}
    \label{alg:kernel}
    \begin{algorithmic}[1]
        \Require{An $n \times m$ matrix $A=(a^1,...,a^n)$, $r > 0 \wedge k \geq 0$}
        \Procedure{Preprocess}{$A,r,k$}
            \State $I \gets \{1,...,n\}$
            \State $L \gets \emptyset$
            \State Construct the partition $\{I_1,...,I_t\}$ of $I$ into initial clusters
            \If{$t>k+r$}
                \State Return NO
            \Else
                \For{$i \gets l$ to $t$}
                    \While{$|I_i| > k+1$}
                        \State Select arbitrary $j \in I_i$
                        \State $I_i \gets I_i \backslash \{j\}$ and $I \gets I \backslash \{j\}$
                    \EndWhile
                \EndFor
                \State $s \gets 0$
                \While{$I \neq \emptyset$}
                    \State $s \gets s+1$
                    \State Select arbitrary $j \in I$
                    \State $S \gets \{j\}$ and $I \gets I \backslash \{j\}$
                    \While{$\exists((p \in S) \cap (q \in I)) \textit{s.t.} d_H(a^p,a^q) \leq k$}
                        \State $S \gets S \cup \{q\}$
                        \State $I \gets I \backslash \{q\}$
                    \EndWhile
                    \State $J \gets \{1,...,m\}$
                    \While{$|J| \geq 2$ and $A[J,S]$ has uniform row with an index $h \in J$}
                        \State $J \gets J \backslash \{h\}$
                    \EndWhile
                    \State $L \gets L \cup A[S,J]$
                \EndWhile
            \EndIf
            \If{$s > r$}
                \State Return NO
            \Else
                \State Return $L$
            \EndIf
        \EndProcedure
    \end{algorithmic}
\end{algorithm}

\subsection{Branching Algorithm}
\begin{algorithm}[H]
    \caption{Binary r-Means}
    \label{main-algo}
    \begin{algorithmic}[1]
        \Require{An $n \times m$ matrix $A=(a^1,...,a^n)$, $r > 0 \wedge k \geq 0$}
        \Procedure{r-Means}{$A,r,k$}
            \State Panic("Not Implemented yet")
        \EndProcedure
    \end{algorithmic}
\end{algorithm}

\subsection{Dynamic programming reconstruction}
The original r-means kernelization algorithm by \cite{fomin_golovach_panolan_2020} 
intended for the sub-matrices $A_i=A_1,...,A_s$ to be stitched together to a single 
matrix before running the r-means algorithm. However as it holds that 
$r_1 + r_2 ... + r_s \leq r\sp{\prime}$ and $k_1 + k_2 ... + k_s \leq k\sp{\prime}$ 
then instead of stitching the solution into a single matrix $A\sp{\prime}$ 
we can run the algorithm multiple times over $A_i$ with increasing parameters $k+r$, 
save the different possible solutions, and stitch the solutions as the end. Thus saving 
memory space and allowing the sub-matrices to be run in parallel.

This reconstruction is done dynamically by constructing a set of possible solutions $S_L$
\begin{algorithm}[H]
    \caption{Solution reconstruction}
    \label{reconstruction}
    \begin{algorithmic}[1]
        \Input{$S$ as a list of lists of hypothesis, $V$ as a a non-empty list of solutions, $B$ as a list of bags}
        \Output{$V$ as the final list of solutions, where $V_k$ is the best solution for $k$. If null, then no solution}
        \Procedure{Reconstruct}{$S, V, B$}
            \ForTo{$s := 1$}{$|S|-1$}
                \State $w \gets S_s$ \Comment{Where $w$ gets the proposed centers}
                \State $s^{\prime} \gets \emptyset$ \Comment{Init proposed solution}

                \ForTo{$i := 0$}{$k$}
                    \State $x \gets i$ and $y \gets 0$ and $p \gets 0$
                    \State $b \gets V_{x} + |w_{y}|$

                    \While{$x > 0$}
                        \State $x \gets x - 1 \wedge y \gets y + 1$
                        \State $c \gets V_{x} + |w_{y}|$

                        \If{$b > c$}
                            \State $b \gets c$
                            \State $p \gets y$
                        \EndIf
                    \EndWhile

                    \State $s^{\prime}_i \gets b$
                    \State $B_i \gets B_i \cup w_p$
                \EndFor
                \State $V \gets s^{\prime}$
            \EndFor

            \State return $V$
        \EndProcedure
    \end{algorithmic}
\end{algorithm}


% % % % % % % % % % % % % % % % %
%
%	4. Implementation
%
\newpage

\section{Algorithm implementation}
\begin{itemize}
    \item No kernel stitching
    \item No "agrees with"
\end{itemize}
\subsection{Tools used}
\subsubsection{The Java programming language, Version 1.6}
Selecting the programming language for any project is always an issue, and the choice of using
the Java programming language was mostly based on availability and the existing operations
and packages in the language, rather than the runtime.
\\
The Java programming language is not as fast as many of its alternatives, but does run almost
seamlessly on any operating system due to running on the JVM (Java virtual machine).

\subsubsection{ArchLinux operating system}
The minimal operating system Linux-distro of ArchLinux was used.

\subsubsection{GCP / BigQuery}
Running several thousand of tests multiple times and storing all results to a database with
input and resulting output can easily fill large amounts of data. To avoid bottlenecking the
testing computer all results could be pushed to an external database. For this the cloud database
"BigQuery" in the "Google Cloud Platform" could be used.

\subsubsection{Google Data studio}
The advantage of using an external database solution like BigQuery is that it's also easily
integrated in other tools. "Google Data studio" can create statistical plots based on large
data systems, and is able to directly query the BigQuery database for almost live updates.

\subsubsection{JetBrains - IntelliJ IDEA}
For any development in a high-level programming language an intelligent development environment 
(IDE) is usually recommended. The IntelliJ IDEA is a premium alternative of many IDEs.

\subsubsection{Git / GitHub}
During development of an application it's recommended to use a version-control tool. GIT is
one of the most known version-control-tools in the field, and allows the user to traverse
the entire solutions history at any time.
\\
To use GIT as a tool, an external GIT hosting solution can improve the experience while acting
as a backup tool. While a simple server could be sufficient, advanced solutions like GitHub 
exist to add extra functionality.

\subsubsection{Testing computer}
The algorithm tests were run on a "Lenovo YOGA 900" with "Intel Core i5 (6. gen) 6200U / 2.3 GHz" 
8GB (1600 MHz / PC3-12800), LPDDR3 SDRAM. 256GB SSD M.2 memory, Intel HD Graphics 520.

\subsection{Data structures}
\subsubsection{BinaryVector}
Each binary vector is represented as a BitSet defined in the Java programming language. A 
BitSet could be defined as a set of positions which is set to $1$. Hence the binary byte 
representation of the number $42$ could be represented as the set 
$42 \equiv 00101010 \equiv (3,5,7)$. The Java library version of BitSet also includes several 
useful binary operations and extras such as cardinality. It should though be noted that the 
BitSet standard is not Thread-safe \cite{bitset_java_14_2020}

An alternative implementation could be to use numbers to store the bits and use the standard 
bit-operators of any programming language. Unfortunately this creates a limit in the length 
of the vectors and their mutability which would require workarounds, but could theoretically 
improve the runtime.

\subsubsection{BinaryMatrix}
The main Binary matrix structure is defined as a list of vectors or BitSets, along with the 
original height and width.

\subsubsection{BinarySubMatrix}
The binary sub matrix is mainly copies of a binary matrix, and along also stores the parent 
binary matrix reference. This allows for deletion of rows and columns, for deleted rows and 
columns to be restored, and to find original positions in the parent matrix.

% % % % % % % % % % % % % % % % %
%
%	5. Testing
%
\newpage

\section{Testing}
\subsection{Generating test matrix}
For the generation of a matrix used for testing we could use any complete randomized binary 
matrix, but to be able to control the result after the completion of the algorithm it is best 
to create more concentrated tests.

\begin{algorithm}[H]
    \caption{Test-Matrix generation}
    \label{alg:test}
    \begin{algorithmic}[1]
        \Require{$d \geq 0 \wedge h > 0$}
        \Procedure{GenerateTestMatrix}{$d,h$}
            \While{$h \geq 0$}
                \State $H_h \gets $ RandomVector
                \State $h \gets h - 1$
            \EndWhile
            \While{$d \geq 0$}
                \State $x \gets $ random vector
                \State $y \gets $ random bit
                \State $H_{xy} \gets \lnot H_{xy}$
            \EndWhile \\
            \Return $H$
        \EndProcedure
    \end{algorithmic}
\end{algorithm}

By reversing the algorithm we can generate a set of vectors $H$, which will be the centres, 
and by picking random vectors from $H$ we will create a matrix with a solution of $K=0$, 
and $R=H$. By then flipping random bits we can slightly modify the expected result, mainly 
the K parameter. By flipping $D$ bits in the matrix we will still have a solvable matrix 
with $R = H$ as long as $K \geq D$ though the matrix might still be solvable within both 
smaller $K$ and $R$.

Still for normal runtime analysis the importance is to test the worst cases for the algorithm, 
and while the concentrated tests is controllable, most of those will be mostly solved by the 
kernel, not allowing for proper testing of the actual algorithm. For this the generation 
algorithm would run with small parameters ($R \leq 5$, and $K \leq 30$).

\subsection{Results}
\subsection{Testing conclusions (Brief 1 paragraph)}

% % % % % % % % % % % % % % % % %
%
%	6. Conclusion
%
\newpage

\section{Conclusion}
\subsection{Results}
\subsection{Future work}
\begin{itemize}
    \item Multithreading
    \item Better data structure and binary operations
\end{itemize}


%%%%%%%%%%%%%%%%%%%%%%%%%%%%%%%%%%%%%%%%%%%%%%%%%%%%%%%%%%%%%%%%%%%%%%%%%%%%%%%%%%%%%%%%%%
%%%%%%%%%%%%%%%%%%%%%%%%%%%%%%%%%%%%%%%%%%%%%%%%%%%%%%%%%%%%%%%%%%%%%%%%%%%%%%%%%%%%%%%%%%



% % % % % % % % % % % % % % % % %
%
%	ADD REFERANCES
%
\newpage
\bibliography{extras/citation-db}
\bibliographystyle{unsrt}
\addcontentsline{toc}{section}{References}

\end{document}
