% !TeX root = pres.tex

\documentclass{beamer}

\usetheme{CambridgeUS}
\usecolortheme{whale}
\usefonttheme{professionalfonts}

\title[Master thesis]{Practical implementation of a parametrized binary matrix clustering algorithm}

\author[Leirvåg, Oskar]
{
    \textit{Student:} Oskar Leirvåg
    \and
    \textit{Mentor:} Petr A. Golovach
}

\institute[UiB] {
  Institute of Informatics: Algorithms \\
  University of Bergen
}

\date[\today]{\today}
%Bookmarks
\usepackage[colorlinks=true,urlcolor=cyan,linkcolor=black,citecolor=red,bookmarksopen=true]{hyperref}
\usepackage{bookmark}

\usepackage{fancyref}
\usepackage[utf8]{inputenc}
\usepackage{amsmath}
\usepackage{amsthm}
\usepackage[framemethod=TikZ]{mdframed}
\usepackage{pgf,tikz}
\usepackage{mathrsfs}
\usepackage{listings}
\usepackage{amssymb}
\usepackage{url}
\usepackage{graphicx}

\usepackage{float}
\floatplacement{figure}{H} % forces figures to be placed at the correct location

\usepackage{enumerate} % Needed for markdown enumerations to work

\usepackage[mathletters]{ucs}

\usepackage{titling} %Remove insane whitespaces

%Margins
\usepackage{geometry}
\geometry{a4paper, margin=3cm}

%Algorithms
\usepackage{algorithm, algorithmicx}
\usepackage{algpseudocode}
\newcommand{\myFigure}[3]{\begin{figure}[h!]\centering\includegraphics[scale=#1]{figures/#2}\caption{#3}\end{figure}}

\newtheorem{theoremdefinition}{\textbf{Definition}}

\algnewcommand\algorithmicforeach{\textbf{foreach}}
\algdef{S}[FOR]{ForEach}[1]{\algorithmicforeach\ #1\ \algorithmicdo}

\algdef{S}[FOR]{ForTo}[2]{\algorithmicfor\ #1 \textbf{to} #2\ \algorithmicdo}

\newenvironment{problem}[2][]{
   \mdfsetup{
        frametitle={
            \tikz[baseline=(current bounding box.east),outer sep=0pt]
            \node[anchor=east,rectangle,fill=white!20]
            {\strut ~#1};}
        }
    
    \mdfsetup{
        innertopmargin=10pt,linecolor=black!30,
        linewidth=1pt,topline=true,
        roundcorner=5pt,
        frametitleaboveskip=\dimexpr-\ht\strutbox\relax
    }
 
\begin{mdframed}[]\relax}{
\end{mdframed}}

\begin{document}
\frame{\titlepage}

\section{Introduction}
\subsection{Clustering}

\begin{frame}
  \begin{block}{Clustering}
    Collecting objects into groups by, \textit{s.t.} each object is \textit{closer}
    to the objects in the same group, than those in other groups, by some measure.
  \end{block}

  In itself, it is mainly a concept, and there exist several different types. Still
  it is the main task in several forms of statistical data-analysis, data mining and
  maybe most commonly known machine-learning.

  Here we will be talking about a specific \alert{centroid-based clustering} algorithm.
\end{frame}

\begin{frame}
  \begin{block}{Centroid-based clustering}
    A form of clustering where we produce or select a single object, not necessarily
    from the original data-set, to represent each cluster.
  \end{block}

  In other terms, instead of producing these groups \textit{(e.g. clusters)}, we will
  from these groups find a \textit{center} or \textit{average}. Thus maybe even
  ignoring the original data from that point on.
\end{frame}

\begin{frame}
  \begin{figure}
    \centering
    \includegraphics[height=0.75\textheight]{figures/1024px-KMeans-density-data.svg.png}
    \caption{By Chire - Own work, CC BY-SA 3.0, \url{https://commons.wikimedia.org/w/index.php?curid=17085333}}
  \end{figure}
\end{frame}

\subsection{Binary data}
\begin{frame}
  The data we will be using is \alert{binary vectors}, which will be
  clustered by their \alert{Hamming distance}, noted by $d_H(x,y)$.

  \begin{block}{Hamming distance}
    The Hamming distance between two binary vectors $x,y \in \{0,1\}^m$
    where $x=(x_1,...,x_m)^T$ and $y=(y_1,...,y_m)^T$, is
    \[
      d_H(x,y)= \sum_{i = 1}^{m} |x_i - y_i|
    \]
  \end{block}
\end{frame}

\begin{frame}
  \begin{block}{Example}
    Given a streaming services data on whether any user likes a given movie,
    we might want to create a recommendation service for new users.
    If we can classify our users into groups that share mostly the same taste in movies, then
    the "average-user" or centres of the classifications would be a "super user" most likely to share the same movie
    interest with any other user in the same classification.
  \end{block}
\end{frame}

\begin{frame}
  From the example, it may be clear that the vectors can represent a
  \alert{binary matrix}, in a clearer way. Thus further we will use columns of binary
  matrices, as the main objects.

  \begin{block}{Binary matrix}
    \[
      A_{m,n} =
      \begin{pmatrix}
        a_{1,1} & a_{1,2} & \cdots & a_{1,n} \\
        a_{2,1} & a_{2,2} & \cdots & a_{2,n} \\
        \vdots  & \vdots  & \ddots & \vdots  \\
        a_{m,1} & a_{m,2} & \cdots & a_{m,n}
      \end{pmatrix}
    \]
  \end{block}
\end{frame}

\subsection{Example}

\begin{frame}
  \begin{block}{$8 \times 8$matrix example}
    \[
      \begin{matrix}
        1 & 0 & 1 & 0 & 1 & 1 & 1 & 1 \\
        0 & 1 & 0 & 1 & 1 & 1 & 1 & 0 \\
        1 & 0 & 1 & 0 & 1 & 1 & 1 & 1 \\
        1 & 1 & 1 & 1 & 1 & 1 & 1 & 1 \\
        0 & 1 & 0 & 1 & 0 & 0 & 1 & 1 \\
        1 & 0 & 0 & 0 & 1 & 1 & 1 & 1 \\
        0 & 1 & 0 & 0 & 0 & 1 & 1 & 1 \\
        0 & 0 & 0 & 0 & 0 & 0 & 0 & 0
      \end{matrix}
    \]
  \end{block}
\end{frame}

\begin{frame}
  \begin{block}{$8 \times 8$matrix example}
    \[
      \begin{matrix}
        1 & 0 & 1 & 0 & 1 & 1 & 1 & 1 \\
        0 & 1 & 0 & 1 & 1 & 1 & 1 & 0 \\
        1 & 0 & 1 & 0 & 1 & 1 & 1 & 1 \\
        1 & 1 & 1 & 1 & 1 & 1 & 1 & 1 \\
        0 & 1 & 0 & 1 & 0 & 0 & 1 & 1 \\
        1 & 0 & 0 & 0 & 1 & 1 & 1 & 1 \\
        0 & 1 & 0 & 0 & 0 & 1 & 1 & 1 \\
        0 & 0 & 0 & 0 & 0 & 0 & 0 & 0
      \end{matrix} \quad \rightarrow \quad \begin{matrix}
        1 & 1 \\
        0 & 1 \\
        1 & 1 \\
        1 & 1 \\
        0 & 1 \\
        0 & 1 \\
        0 & 1 \\
        0 & 0
      \end{matrix}
    \]
  \end{block}
\end{frame}

\section{Problem statement}

\subsection{Formal definition}
\begin{frame}
  \begin{block}{Binary r-Means}
    \begin{tabular}{p{0.1\textwidth}p{0.8\textwidth}}
      \textit{Input}: & An $n \times m$ binary matrix \textbf{A} with columns
      ($\textbf{a}^1,...,\textbf{a}^n$), a positive integer $\textbf{r}$ and a nonnegative
      integer $\textbf{k}$                                                                     \\

      \textit{Task}:  & Decide whether there is a positive integer $\textbf{r}\sp{\prime} \leq
        \textbf{r}$, a partition $\{I_1, ..., I_{r\sp{\prime}}\}$ of indices $\{1,...,n\}$ and vectors
      $(\textbf{c}^1,...,\textbf{c}^{r\sp{\prime}}) \in \{0,1\}^m$ such that

      \[
        \sum_{i = 1}^{r\sp{\prime}} \sum_{j \in I_i} d_H(c^i, a^j) \leq k
      \]
    \end{tabular}
  \end{block}
\end{frame}

\subsection{Task and motivation}
\begin{frame}
  \begin{block}{Task}
    Not many algorithms for clustering binary vectors exactly. And due to the
    nature of the existing algorithms, their practical implementation should be tested
  \end{block}
\end{frame}

\section{The algorithm}
\begin{frame}

\end{frame}

\end{document}